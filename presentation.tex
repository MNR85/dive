% compiling: pdflatex -shell-escape dive.tex

\documentclass[aspectratio=43]{beamer}
\usetheme{Berkeley}

\usepackage{color}
%\usepackage{graphicx}
\usepackage{amsmath}
\usepackage{soul}
\usepackage{dot2texi}
\usepackage{tikz}
\usepackage{color,listings}
\usetikzlibrary{shapes,arrows}

\lstset{
    numberstyle=\ttfamily,
    basicstyle=\ttfamily,
    breaklines=true,
    postbreak=\raisebox{0ex}[0ex][0ex]{\hspace{30pt}\ensuremath{\color{red}\hookrightarrow\space}}
}
\newcommand{\G}[1]{\colorbox{lightgray}{#1}}
\newcommand{\C}{\lstinline}

\title{Dive}
\author{Vitaly\ \"\_Vi\"\ Shukela}
\date{July 21, 2015}

\AtBeginSection[] {
\begin{frame} 
    \frametitle{Presentation Outline}
    \tableofcontents[currentsection]
    \end{frame}
}

\begin{document}

\begin{frame}
    \titlepage
\end{frame}

\section{Initial rationale}

\begin{frame}
    \frametitle{History}
    I was experimenting with LXC containers and found out that after stating the container it is tricky to launch some additional program into it.
    \pause\\\vfill
    The supposed way of doing this was configuring the network starting SSH server and using it:
\end{frame}

\begin{frame}[fragile]
    \frametitle{LXC and sshd}
    \begin{dot2tex}[autosize,scale=0.7] digraph G {
        Terminal -> LXCExecute -> bash -> sshd
        Terminal2 -> ssh
        sshd -> bash2
        ssh -> sshd [style=dashed]
        LXCExecute [label="lxc-execute"]

        subgraph cluster_i {
            label="Inside container"
            sshd
            bash
            bash2
        }
    } \end{dot2tex}
\end{frame}

\begin{frame}
    \frametitle{History}
    And I wanted this:
\end{frame}


\begin{frame}[fragile]
    \frametitle{Direct}
    \begin{dot2tex}[autosize,scale=0.8] digraph G {
        Terminal -> LXCExecute -> bash
        Terminal2 -> bash2
        LXCExecute [label="lxc-execute"]

        subgraph cluster_i {
            label="Inside container"
            bash
            bash2
        }
    } \end{dot2tex}
\end{frame}

\begin{frame}
    \frametitle{History}
    I wanted to do it:
    \begin{itemize}
        \item Without using virtual network;
        \item Without heavyweight additional programs;
        \item Preserving all FDs, not just stdin/stdout/stderr;
    \end{itemize}
\end{frame}

\section{Original dive}

\begin{frame}
    \frametitle{Initial dive}
    So I implemented a program that listens a socket and allows remotely starting programs:
    \begin{itemize}
        \pause \item No TCP, only UNIX socket;
        \pause \item Using sendfd/recvfd;
        \pause \item No authentication, but preseving user (\C!SO_PEERCRED!);
        \pause \item Preserving signals;
        \pause \item Passing command line and environment variables as array;
        \pause \item Preserving controlling terminal (1/2);
        \pause \item \st{Preserving process parents} (failed);
    \end{itemize}
\end{frame}

\begin{frame}
    \frametitle{Initial dive}
    So, initial dive rationale is \\
    \vfill
    \Large{Poor man's SSHd for starting things inside lxc-execute}.
\end{frame}

\section{Updated rationale}

\begin{frame}
    \frametitle{History}
    I finished playing with LXC at that moment, but used "dive" project as playground.\\
    \vfill
    \pause More features creeped in, so I created "nocreep" branch in Git to preserve "poor man's sshd" dive as a little program.
\end{frame}

\begin{frame}
    \frametitle{Updated rationale}
    The new rationale is:\\
    \vfill
    \Large{Be a tool for starting processes in various ways, like \emph{socat} is the tool for using sockets.}
\end{frame}

\section{Usage example: restrict suidbit}


\begin{frame}
    \frametitle{\C!PR_SET_NO_NEW_PRIVS!}
    I don't like suidbit feature.
    \pause  \vfill 
    I want to start a program that should not be able to elevate it's privileges by filesystem means.
    \pause \vfill
    \G{\C!dived -J -S -T -P    -X    -- ./some_program arguments!}
\end{frame}

\section{Usage example: sudo}

\begin{frame}
    \frametitle{suid-less sudo}
     I don't like suidbit feature.
     \pause  \vfill I want to run large part of the system with suidbit forbidden (\C!-o nosuid!, \C!PR_SET_NO_NEW_PRIVS!).
     \pause  \vfill But also want this part to elevate privileges in a controlled way.
\end{frame}


\begin{frame}
    \frametitle{access to one root program for a specific user}
    Let's give someuser access to run some\_program\_only as root without using setuid.
    \pause \vfill
    \G{\C!dived /home/someuser/poormansudo -d -C 700 -U someuser:someuser -E -P -- /usr/bin/some_program_only!}
    \pause \vfill
    \G{\C!dive ./poormansudo --some --arguments!}
\end{frame}

\begin{frame}
    \frametitle{give anybody chroot, but revoke setuid magic}
    
    I like to use \C!chroot! for development and don't want to change to root every time.

    \pause \vfill
    \G{\C!dived /var/run/chrooter -d -C 777 -X -r!}
    \pause \vfill
    \G{\C!DIVE_ROOTDIR=/home/user/system dive /var/run/chrooter /bin/bash !}
\end{frame}

\section{Usage example: unshare}

\begin{frame}
    \frametitle{namespace handling}
    Let's become poor man's LXC.
    \pause \vfill
    Start a "container":\\
    \G{\C!dived -J -S -T -P  -s net,fs -- /bin/bash!}
    \pause \vfill
    Start additional program into that container:\\
    \G{\C!dived -J -S -T -P  -N /proc/12345/ns/net -N /proc/12345/ns/mnt -- /bin/bash!}
\end{frame}

\section{Feature list}

\begin{frame}
    \frametitle{Feature list}
    \begin{itemize}
        \item Starting programs directly ...
        \item ... or initiated by socket
        \hspace{20pt} \begin{itemize} 
            \item Preservation of argv and envp arrays
            \item Preservation of controlling terminal (limited)
            \item Preservation of uid/gid (initializing other groups)
            \item Signal preservation
            \item Waiting for termination of a remotely started process
            \item inetd mode
            \item Abstract sockets
        \end{itemize}
        \item Capability, securebits and \C!PR_SET_NO_NEW_PRIVS! management
        \item Namespace management
        \item "authenticate" feature
        \item Resource (rlimit) management
        \item Creation of pidfile
        \item Chrooting
        \item Saving of pidfile
    \end{itemize}
\end{frame}

\begin{frame}
    \frametitle{Current omissions}
    TODO:
    \vfill
    \begin{itemize}
        \item Sane command line argument handling
        \item Full-coverage tests
        \item Cgroups management
        \item Distribution package inclusions
        \item Refactoring
    \end{itemize}
\end{frame}

\section{Conclusion}

\begin{frame}
    Dive is a project that helps to start programs in a light-weight, but versatile way.\\
    \vfill
    \C!https://github.com/vi/dive!\\
    \vfill
    \centerline{\LARGE{The end.}}
\end{frame}

\end{document}

